\chapter*{Resumen}

Una \textit{foliación holomorfa singular por curvas} es una estructura geométrica definida sobre una variedad compleja, cuyo prototipo local es la familia de curvas integrales de un campo vectorial holomorfo. Los ceros de estos campos locales, denominados \textit{puntos singulares} de la foliación, son especiales tanto desde un punto de vista topológico como analítico, ya que la curva integral que pasa por un punto singular es simplemente el punto singular mismo. En este trabajo, contaremos los puntos singulares de una foliación por curvas de una variedad compleja compacta.

Pese a la naturaleza geométrica de nuestro problema, la principal herramienta que usaremos para resolverlo es la topología algebraica. Más precisamente, construiremos las \textit{clases de Chern} $c_i(E)$ de un fibrado vectorial complejo $E \to M$ y las interpretaremos como \textit{obstrucciones} a que existan una o varias secciones linealmente independientes de $E$. Aplicando esta interpretación a una variedad compleja compacta $M$ y un fibrado tangente torcido $E = TM \otimes L$, obtendremos el número de puntos singulares de una foliación definida por una sección holomorfa de $E$.