\section{La primera obstrucción}

La \textbf{primera obstrucción} a la existencia de secciones globales de $V_k(E)$ o, por brevedad, la primera obstrucción de $V_k(E)$, se define como la clase de cohomología celular
$$c \circ V_k(E) = [c(\sigma)] \in H^{2q+2}(M)$$
del cociclo de obstrucción $c(\sigma)$ de cualquier sección $\sigma : M^{2q+1} \to V^{2q+1}$. Por los resultados de la sección anterior, la primera obstrucción $c \circ V_k(E)$ tiene las siguientes propiedades.

\begin{proposition}[Naturalidad]
Dada una función continua $f : N \to M$ cuyo dominio es otro complejo celular $N$, la primera obstrucción del fibrado pullback $f^\star \circ V_k(E)$ es
$$c \circ f^\star \circ V_k(E) = f^\star \circ c \circ V_k(E) \in H^{2q+2}(N).$$
\end{proposition}

\begin{proof}
Reemplacemos a $f$ con una aproximación celular \cite[p. 349]{hatcher1}. Por la proposición 5.4, dada una sección arbitraria $\sigma : M^{2q+1} \to V^{2q+1}$, tenemos
$$
c \circ f^\star \circ V_k(E)
    = [c \circ f^\star(\sigma)]
    = [f^\star \circ c(\sigma)]
    = f^\star([c(\sigma)])
    = f^\star \circ c \circ V_k(E).
$$
Por lo tanto, la primera obstrucción es natural con respecto a los pullbacks.
\end{proof}

\begin{proposition}[Estabilidad]
La primera obstrucción de $V_{k+j}(E \oplus \C^j)$ es
$$c \circ V_{k+j}(E \oplus \C^j) = c \circ V_k(E).$$
\end{proposition}

\begin{proof}
Sea $\sigma : M^{2q+1} \to V^{2q+1}$ una sección arbitraria. Por la proposición 5.5,
$$
c \circ V_{k+j}(E \oplus \C^j)
    = [c \circ i_\star(\sigma)]
    = [i_\star \circ c(\sigma)]
    = i_\star([c(\sigma)])
    = i_\star \circ c \circ V_k(E)
    = c \circ V_k(E),
$$
donde, en el último paso, hemos usado el hecho de que el isomorfismo canónico
$$i_\star : \pi_{2q+1} \circ V_k(\C^n) \longrightarrow \pi_{2q+1} \circ V_{k+j}(\C^{n+j})$$
se representa por el automorfismo identidad de $\Z$. Por lo tanto, la primera obstrucción es estable con respecto a la adición de sumandos directos triviales.
\end{proof}

\begin{theorem}
La condición $c \circ V_k(E) = 0$ es necesaria y suficiente para que exista una sección de $V_k(E)$ sobre el $(2q+2)$-esqueleto de $M$.
\end{theorem}

\begin{proof}
Dada una sección parcial $\tau : M^{2q+2} \to V^{2q+2}$, la restricción $\sigma = \tau \big \vert_{M^{2q+1}}$ es extendible a una sección de $V^{2q+2}$, a saber, $\tau$. Entonces, $c \circ V_k(E) = [c(\sigma)] = [0] = 0$.

Recíprocamente, si $c \circ V_k(E) = 0$, entonces el cociclo de obstrucción $c(\sigma)$ de cualquier sección parcial $\sigma : M^{2q+1} \to V^{2q+1}$ es una cofrontera. Por el teorema 5.7, mediante una homotopía hueca $\rho : M^\square \to V^\square$ que comienza en $\sigma$, podemos obtener otra sección $\widetilde \sigma : M^{2q+1} \to V^{2q+1}$ cuyo cociclo de obstrucción es $c(\widetilde \sigma) = 0$. Entonces, $\widetilde \sigma$ se extiende a una sección de $V^{2q+2}$.
\end{proof}

Las propiedades de la primera obstrucción $c \circ V_k(E)$ antes demostradas son sospechosamente parecidas a uno de los axiomas de las clases de Chern (naturalidad) y a una de sus consecuencias inmediatas (estabilidad). A continuación, demostraremos que esto no es coincidencia: la primera obstrucción $c \circ V_k(E)$ es precisamente la $(q+1)$-ésima clase de Chern $c_{q+1}(E)$.

\begin{proposition}
La \textbf{primera obstrucción universal}\footnote{El grassmanniano $\Gr_n(\C^\infty)$ tiene una descomposición celular canónica en \textit{células de Schubert}.}
$$c_{k,q} = c \circ V_k \circ E_n(\C^\infty) \in H^{2q+2} \circ \Gr_n(\C^\infty)$$
determina el valor de $c \circ V_k(E)$ para cualquier fibrado vectorial $E \to M$ de rango $n$ cuyo espacio base $M$ es un complejo celular.
\end{proposition}

\begin{proof}
Análoga a la proposición 4.4.
\end{proof}

\begin{proposition}
La primera obstrucción universal es de la forma
$$c_{k,q} = \lambda_{q+1} \, c_{q+1}^n,$$
donde $\lambda_{q+1} \in \Z$ es una constante que sólo depende de $q$.
\end{proposition}

\begin{proof}
Sea $f : \Gr_q(\C^\infty) \to \Gr_n(\C^\infty)$ una función clasificadora de $E = E_q(\C^\infty) \oplus \C^k$. Dado que $V_k(E)$ posee una sección global inducida por la base estándar de $\C^k$, tenemos
$$c_{k,q} \in \ker(f^\star) = \langle c_{q+1}^n \rangle.$$
Como $c_{k,q}$ y $c_{q+1}^n$ son del mismo grado, existe una constante entera $\lambda \in \Z$ tal que
$$c_{k,q} = \lambda c_{q+1}^n.$$
Notemos que, en principio, la constante $\lambda$ depende tanto de $k$ como de $q$. Sin embargo, dado que la primera obstrucción es estable, tenemos
$$c_{1,q} = c_{2,q} = \dots = c_{k,q} = \lambda c_{q+1}^n.$$
Por lo tanto, $\lambda = \lambda_{q+1}$ sólo depende de $q$.
\end{proof}

\begin{theorem}
La primera obstrucción universal es
$$c_{k,q} = c_{q+1}^n \in H^{2q+2} \circ \Gr_n(\C^\infty).$$
Por ende, la $(q+1)$-ésima clase de Chern $c_{q+1}(E)$ es la primera obstrucción a la existencia de $k$ secciones de $E \to M$ linealmente independientes sobre cada punto $p \in M$.
\end{theorem}

\begin{proof}
A partir de la sucesión exacta de Euler \cite[pp. 93-94]{huybrechts}
$$
\begin{tikzcd}[row sep = large, column sep = large]
    0 \arrow[r] & \O_{\P^n} \arrow[r] & \O_{\P^n}(1)^{n+1} \arrow[r] & T\P^n \arrow[r] & 0,
\end{tikzcd}
$$
deducimos lo siguiente:
\begin{itemize}
    \itemsep 0em
    \item Todo campo vectorial holomorfo global sobre $\P^n$ es de la forma
    $$X = f_0 \, \p {x_0} + \dots + f_n \, \p {x_n},$$
    donde $f_0, \dots, f_n \in \O_{\P^n}(1)$ son formas lineales.
    
    \item El operador lineal $f = (f_0, \dots, f_n)$ que representa a $X$ está bien definido salvo un múltiplo constante del operador identidad.
\end{itemize}
Notemos que $f$ tiene las siguientes propiedades:
\begin{itemize}
    \itemsep 0em
    \item $[v] \in \P^n$ es una singularidad de $X$ $\iff$ $v$ es un autovector de $f$.
    \item $[v] \in \P^n$ es una singularidad aislada de $X$ $\iff$ $\langle v \rangle$ es un autoespacio de $f$.
    \item $[v] \in \P^n$ es una singularidad simple de $X$ $\iff$ $\langle v \rangle$ es un autoespacio generalizado de $f$.
\end{itemize}
En particular, si $X$ un campo holomorfo general sobre $\P^n$, entonces
\begin{itemize}
    \itemsep 0em
    \item $X$ tiene $n+1$ singularidades simples $p_0, \dots, p_n$.
    \item Dada una descomposición celular de $\P^n$ con una $2n$-célula distinguida $e_i$ centrada en cada punto singular $p_i$, el cociclo de obstrucción de $X$ es
    $$
    c(X)(e_\alpha) =
        \begin{cases}
            1, & \text{para $e_\alpha = e_0, \dots, e_n$,} \\
            0, & \text{para cualquier otra $2n$-célula.}
        \end{cases}
    $$
    
    \item La primera obstrucción a la existencia de un campo global sin ceros sobre $\P^n$ es
    $$c \circ V_1(T\P^n) = [c(X)] = (n+1) \, \xi^n = c_n(T\P^n),$$
    donde $\xi \in H^2(\P^n)$ es la clase del hiperplano.
\end{itemize}
Esto sólo es posible si $\lambda_n = 1$, lo cual implica el resultado buscado.
\end{proof}
