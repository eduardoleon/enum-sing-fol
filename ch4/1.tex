\section{Definiciones básicas}

Por el teorema de clasificación del capítulo anterior, la estructura de un fibrado vectorial de rango $n$ sobre un espacio topológico paracompacto $M$ está plasmada \textit{sin pérdida de información} en una función clasificadora $f : M \to \Gr_n(\C^\infty)$. Sin embargo, en general, esta función es difícil de construir y aún más difícil de usar, al menos para resolver el tipo de problemas geométricos que motiva nuestro estudio de los fibrados vectoriales.

Dado que no necesitamos toda la información contenida en una función clasificadora, estamos interesados en hallar un objeto más simple que rescate la información más esencial de un fibrado vectorial $E \to M$. En este capítulo, construiremos una sucesión de clases de cohomología
$$c_i(E) \in H^{2i}(M),$$
llamadas las \textbf{clases de Chern} de $E \to M$, que describen el homomorfismo de anillos
$$f^\star : H^\star \circ \Gr_n(\C^\infty) \longrightarrow H^\star(M)$$
inducido por una función clasificadora $f : M \to \Gr_n(\C^\infty)$ de $E$. La suma formal
$$c(E) = c_0(E) + c_1(E) + c_2(E) + \dots$$
también se denomina la \textbf{clase total de Chern} de $E$.

Las clases de Chern están completamente determinadas por tres axiomas:

\begin{axiom}[Naturalidad]
$c(F) = f^\star \circ c(E)$, para todo fibrado pullback $F \cong f^\star(E)$.
\end{axiom}

\begin{axiom}[Fórmula de Whitney]
$c(E) = c(F) \smile c(E/F)$, para todo subfibrado $F \subset E$.
\end{axiom}

\begin{axiom}[Normalización]
$c \circ E_1(\C^\infty) = 1 - \xi$, donde $\xi$ es el generador\footnote{El isomorfismo canónico $H^\star(\P^\infty) \cong \Z[\xi]$ se calcula construyendo el espacio proyectivo $\P^\infty = \Gr_1(\C^\infty)$ como el límite directo del sistema de inclusiones $\P^0 \to \P^1 \to \P^2 \to \dots$, con la topología celular.} de $H^\star(\P^\infty)$.
\end{axiom}

Las clases de Chern de un fibrado vectorial $E \to M$ se pueden construir de varias formas. En este capítulo, usaremos el método que requiere menos esfuerzo, que consiste en usar los axiomas para constreñir los valores posibles de $c_i(E)$ hasta que sólo quede una opción.

\begin{proposition}
Las clases de Chern $c_i(E)$ de un fibrado vectorial complejo arbitrario $E \to M$ de rango $n$ están determinadas por las \textbf{clases de Chern universales}
$$c_i^n = c_i \circ E_n(\C^\infty) \in H^\star \circ \Gr_n(\C^\infty)$$
asociadas al fibrado tautológico $E_n(\C^\infty) \to \Gr_n(\C^\infty)$.
\end{proposition}

\begin{proof}
Por el axioma de naturalidad, la $i$-ésima clase de Chern de $E$ es
$$c_i(E) = c_i \circ f^\star \circ E_n(\C^\infty) = f^\star(c_i^n),$$
donde $f : M \to \Gr_n(\C^\infty)$ es cualquier función clasificadora de $E$.
\end{proof}

Por consistencia con la notación para la clase total $c(E)$, escribiremos
$$c^n = c_0^n + c_1^n + c_2^n + \dots = c \circ E_n(\C^\infty).$$
Notemos que el axioma de normalización especifica directamente el valor de $c^1$. Para calcular las clases $c^n$ en general, usaremos el concepto de \textbf{rompimiento} definido en la siguiente sección.
