\section{Conclusiones}

En este trabajo, hemos usado las clases de Chern para hallar el número de singularidades de una foliación holomorfa singular por curvas $\F$ tres una variedad compleja compacta $M$. Para ello, hemos demostrado cuatro hechos importantes:

\begin{enumerate}[label=\alph*)]
    \itemsep 0em
    \item Que la definición geométrica de foliación holomorfa (en términos de cartas distinguidas) se puede reexpresar en términos de secciones de un fibrado tangente torcido $TM \otimes L$.
    
    \item Que el número de Milnor $\mu_p(\F)$ es la noción natural de multiplicidad de un punto singular $p \in \Sing(\F)$, por razones tanto topológicas como analíticas.
    
    \item Que las clases de Chern $c_i(E)$ de un fibrado vectorial complejo $E \to M$ son obstrucciones a la existencia de secciones continuas linealmente independientes $M \to E$.
    
    \item Que las clases de Chern $c_i(E)$ son relativamente fáciles de calcular, sobre todo cuando $E$ se relaciona con otros fibrados vectoriales mediante operaciones tales como la suma directa, el producto tensorial y el pullback.
\end{enumerate}

Una limitación de la manera como hemos usado las clases de Chern es que, hasta ahora, sólo hemos enumerado las singularidades de $\F$ cuando $\Sing(\F)$ es un conjunto discreto. Si $\Sing(\F)$ es de dimensión positiva, entonces necesitamos apelar al concepto algebro-geométrico de \textbf{exceso de intersección}, que, no obstante, trasciende el alcance de este trabajo. El lector puede encontrar más información sobre este tema en \cite[cap. 9, pp. 153-174]{fulton}.
